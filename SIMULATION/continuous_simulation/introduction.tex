\section{Introduction}
Tasks such as predicting the future state of a manufacturing line, analyzing the effect of the line parameters and optimization of the buffer need a model that can provide the key indicators such as throughput of a line and average buffer levels. Analytical models such as decomposition were popular and the fastest way to analyze the effect of the line parameters on these key indicators. However, they come with a lot of assumptions and this will lead to simplification of the line, this renders an inaccurate prediction in case of complex lines. This gap has been filled by using simulation models, unlike analytical model simulations can replicate the exact behavior until to the scale of user need. Discrete event simulation is one such method to simulate a system. Depending on the scale and parameters in the simulation, DES can be painfully slow. This makes it a difficult choice during optimization or any other application where we need to perform many simulations. The slowness of a DES simulation is inherited from the level of abstraction at which the events/operation are defined in simulation. For example, consider a transfer line where machines are separated by a finite sized buffer. One can extract the events to be simulated into two sets. \par
\begin{itemize}
    \item High frequency operation (operation on product, exit and entry of parts at machines). 
    \item Low frequency operation (machine failure and repairs, blocking or starvation).
\end{itemize}

A DES model is designed to simulate both these event groups, thus making the algorithm slower. Other than simulating the part as a discrete element, one can approximate the flow of the material to be continuous. This will help us not to visit the state where a part enters a machine and leaves a machine. This is the idea of a continuous simulation (CS). We approximate a discrete part system as a continuous material system by only simulating the low frequency high time period events such as machine failure, repair, blocking and starvation. This idea can be found in literature, the most recent advancements for CS are presented in the work of Salvatore Scrivano et al \cite{scrivano2023continuous}. They presented a generalized CS model that can handle from transfer lines to loops, they reported CS in general is about 15 times faster than DES. However, the proposed model was only for single part-time and is not flexible enough to apply a user control policy. Also, it lacks some robust testing methodology to compare the results acquired from CS with respect to DES. Moreover, in an algorithm during a starvation or blocking the processing rates of the upstream and downstream machines are modified according to a parameter called characteristics' length that is evaluated as the longest chain length in the given graph. This results in approximations on the processing rates of the machines. In our article we replace this method and evaluate the processing rates of the machines based on the upstream and downstream of the machines. This will help us find the exact processing rates for the machines. \par
The conclusions and observations drawn from this literature review, helped us to formulate a problem statement that will help us to develop a generalized continuous simulation model. \par
\begin{itemize}
    \item Build a generalized model to handle lines, trees and loops.
    \item Evaluate the exact state of the simulation using the information from the graph.
    \item Handle the complex flow systems such as multiple part types. 
\end{itemize}
In this report is a first approach from Dillygence to address the statements that were mentioned above. We will present a CS algorithm to solve the transfer lines. The later part of the document you will see the introduction to transfer lines and nomenclature, followed by a continuous simulation algorithm and finally ends with the results and conclusions. \par
