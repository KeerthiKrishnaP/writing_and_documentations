\section{Abstract}
Simulation has been used to analyze, optimize the performance of complex manufacturing systems.
Since simulations are more flexible than the real world system, it can be a very sophisticated 
choice for generating data that can be used in data hungry methodologies like machine learning (ML). 
Discrete event simulation (DES) is a standard computational method to evaluate the state space of a system, 
which requires a complete description of all the events (low and high frequency operations / events)
on the shop floor that were executed in series. This results in an algorithm that is inherently slow. 
On the other hand, approximating a discrete event system to a continuous model will allow us to jump 
from state space to the next without simulating all the operations. In this work we developed a robust 
continuous simulation model that can handle transfer lines and trees. This study also contains a rigorous 
comparison between the developed model against the results from discrete events simulation and analytical model. 
During this we studied the effect of convergence criterion on the steady state prediction of the model.\par
\vspace{5pt}
\textit{Keywords: Manufacturing systems, Simulation, Optimization, Discrete event simulation, Continuous simulation.}\par 
\noindent\hrulefill \par
\vspace{5pt}
\vspace{5pt}
\textbf{Version FR} \par
\vspace{5pt}
La simulation a été utilisée pour analyser et optimiser les performances de systèmes de fabrication complexes.
Comme les simulations sont plus flexibles que le système réel, elles peuvent constituer un choix très 
sophistiqué pour générer des données qui peuvent être utilisées dans des méthodologies avides de données 
telles que l'apprentissage automatique (ML). La simulation d'événements discrets (DES) est une méthode de 
calcul standard pour évaluer l'espace d'état d'un système, qui nécessite une description complète de tous 
les événements (opérations/événements à basse et haute fréquence) de l'atelier qui ont été exécutés en série. 
Il en résulte un algorithme intrinsèquement lent. D'autre part, l'approximation d'un système à événements discrets 
par un modèle continu nous permet de passer d'un espace d'état à l'autre sans simuler toutes les opérations. 
Dans ce travail, nous avons développé un modèle de simulation continu robuste qui peut gérer les lignes de 
transfert et les arbres. Cette étude contient également une comparaison rigoureuse entre le modèle développé 
et les résultats de la simulation d'événements discrets et du modèle analytique. Au cours de cette étude, 
nous avons étudié l'effet du critère de convergence sur la prédiction de l'état stable du modèle.\par
\vspace{5pt}
\textit{Mots-clés : Systèmes de fabrication, Simulation, Optimisation, Simulation d'événements discrets, Simulation continue.}

